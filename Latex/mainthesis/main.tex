\documentclass[12pt,a4paper]{article}
\usepackage[abbreviate = true, doi = true, style = nature, giveninits = true, sorting = none, backend = biber]{biblatex}
\usepackage[lmargin = 3.5cm, rmargin = 3.5cm, tmargin = 2.5cm, bmargin = 2.5cm]{geometry}
\usepackage[onehalfspacing]{setspace}
\usepackage{graphicx}
\usepackage{hyperref}
\usepackage{times}
\usepackage{csquotes}
\usepackage[UKenglish]{babel}
\usepackage[textsize=tiny]{todonotes}
\usepackage[acronym]{glossaries}
\usepackage{soul} % for command \hl
\usepackage{pdfpages} % to insert pdf docs
\usepackage{csvsimple} % to import .csv files as tables
\usepackage{siunitx}
\usepackage{tikz}
\usepackage{caption} 
\captionsetup[table]{skip=10pt} % more space between table and caption

\AtBeginBibliography{\small}
\addbibresource{main_sources.bib}

\setlength{\parindent}{0cm}
\setlength{\marginparwidth}{3.5cm} % make todonotes wider
\newcommand{\todoleft}[1]{{\reversemarginpar \todo{#1}}}
% preset for italic species name, abbreviated after first use
\newacronym[first = {\textit{Harmonia axyridis}}]{harm}{\textit{H. axyridis}}{\textit{Harmonia axyridis}}
\glsdisablehyper % no hyperref for \gls


%%%%%%%%%%%%%%%%%%%%%%%%%%%%%%%%%%%%%%%%%%%%%%%%%%%%%%%%%%%%%%%%%%
\begin{document}
\def\findate{\today}

%einreichung-der-bachelorarbeit
\includepdf[pages=-]{einreichung_bachelorarbeit.pdf}
%Einverstaendnis mit Veroeffentlichung der Bachelorarbeit
\thispagestyle{empty}
I, Lukas Prader, agree that my bachelor's thesis ("Title"), supervised by Dr. Lauren Talluto, will be digitally archived in a repository of Leopold Franzens University Innsbruck and made available for further academic use. \\

Innsbruck, \findate					\hfill Signature

%title page
\newpage
\thispagestyle{empty}
\begin{center}
    \Large{University of Innsbruck \\ Faculty of Biology} \\
    \vspace{3mm}
    \large{Department of Ecology}
    \vspace{10mm}

    \includegraphics[width = 0.6 \linewidth]{universitaet-innsbruck-logo-cmyk-farbe.jpg}

    \vspace{10mm}
    \Large{Bachelor Thesis} \\
    \large{submitted for the degree of} \\
    \Large{Bachelor of Science} \\
    \vspace{10mm}
    \LARGE{\textbf{Studying SDM performance throughout a time series: A case study using the invasive species \textit{Harmonia axyridis}}} \\
    \vspace{10mm}

    \large{by \\ Lukas Prader \\ Matriculation Nr.: 12115058 \\ SE Biological Seminar: Ecology}
\end{center}

\vspace{30mm}
\begin{tabular}{ll}
    \large{Submission Date:} & \large{\findate}                       \\
    \large{Supervisors:}     & \large{Lauren Talluto, Gabriel Singer} \\
\end{tabular}

\newpage
\thispagestyle{empty}
\begin{abstract}
    \noindent In invasive ecology, a lot of research is aimed at predicting the potential impact that a species could have outside its native range.
    Species Distribution Models (SDMs) have been applied in this context already, but with still ongoing discussion of their actual viability given the dynamic aspect of niches in the process of invasion.
    To further insight into the applicability of SDMs for predicting invasive species, a SDM ensemble time series modelling the European niche of \textit{Harmonia axyridis} over the time period 2002-2020 was created.
    In addition, niche dynamic analyses were conducted, computing the niche stability and expansion over time among other parameters.
    It was shown that SDMs trained on data only from the invaded range provide consistently high TPR (True Positive Rate), and this in the context of an almost completely stable niche over the whole time period.
    Models with data only from the invaded range, as well as models using native data only are consistent with previous work predicting the range of \textit{H. axyridis}. \\

    \noindent Keywords: Species Distribution Model, invasive species, ensemble model, time series, niche dynamic analysis, ecospat
\end{abstract}

\newpage
\tableofcontents
\thispagestyle{empty}
\newpage
\pagenumbering{arabic}

\section{Introduction} \label{sec:introduction}
% Invasion Theory
Invasive species are of special interest in ecological research due to their impact on native ecosystems.
Main goals in this area are to find out which species have potential to become invasive, what habitat will be susceptible to invasion by those species, how fast the species will invade the new area and what impact its invasion will have on the native ecosystem \autocite{shigesada1997invasions}.
To this end, many theories have been created to describe invasion processes.
The invasion of a species can  generally be described with four stages \autocite{blackburn2011invasionstages}:
\begin{enumerate} % just have keywords in a sentence instead of enumeration?
    \item Transport: Leaving the native range, arriving at a new location
    \item Introduction: Existing in specific locations (captivity / cultivation)
    \item Establishment: Existing outside of areas of introduction in the wild
    \item Spread: Sustaining establishment and dispersing to new environments
\end{enumerate}
Depending on the current stage there can be significant differences in behaviour and impact of a species.
The impacts of invasive alien species can be numerous, ranging from food web changes to reductions in habitat and species richness, hydrology and nutrient cycle changes, enhanced invasion of other species and mass extinctions \autocite{simberloff2013invasiveimpacts}.
For example intraguild predation, the predation of species using similar resources, can create completely new stable states of an ecological system \autocite{polis1989theoryIGP}.
Fully understanding the dynamics at play during the invasion process would open more possibilities to actively influence the invasion of threatening species.
For this, creating models which are able to predict the invasion is one current focus of research.
Since invasion theory already uses niche theory, it is quite appropriate to think about applying niche models to the problem.

% SDM Theory
Species Distribution Models (SDMs), are being applied to predict the further development of species occurrences in many contexts, also for invasions.
These types of models have been shown to generate substantial insight into the ecological requirements of species and, as niche models, can be used to predict the potential habitat of a species \autocite{araujo2006sdmchallenges}.
There has been considerable debate on the capabilities and limitations of SDMs, especially when used for prediction outside the data domain.
In general, SDMs are made with the (ideal) assumption that the species is in environmental equilibrium \autocite{elith2009sdmtheory}, implying that its ecological niche is not currently changing.
If these models are now used to predict new, unsampled areas, there actually is no measure to assess their accuracy, since no data is presently available for that area \autocite{araujo2006sdmchallenges}.
This means that when trying to predict areas which are potentially outside the calibration range, sufficient validation data is lacking, implying strong uncertainty about the predictive performance of a given model \autocite{araujo2006sdmchallenges}.
This issue of model transferability is an ongoing area of research in the SDM community.
There is also no guarantee that the biotic interactions sampled in the study area will reflect the final interactions in the new area \autocite{elith2009sdmtheory}.
All of these issues apply especially to the prediction of invasive species, since there might be limited data in the invaded range, the species is often not currently at equilibrium and interactions with native species are completely new \autocite{mainali2015sdmprojecting}.
Despite all these challenges, SDMs have been used numerous times to provide insight into the invasive potential and the invasion dynamics of alien species \autocite{zimmermann2010sdmtrends}.
One way of gaining more insight into the invasion process is to create models with data from different time periods during the invasion \autocite{briscoe2019palmerisdm}.
For example, data from a time period early in the invasion process can be used to build models which then are evaluated against data from a later time period \autocite{barbet2018nigrithoraxsdm}.
With this, SDMs can be used to detect niche shifts, which in turn improves the understanding of the underlying niche dynamics and their impact, which helps to put model performance into perspective, for example when using its results for risk assessment of potential invasions \autocite{pearman2008nicheSDM}.

% Modelling Method
SDM performance is not only influenced by the underlying data, but also the type of model chosen for the analysis.
Models range from regression methods to machine learning and each feature various strengths and weaknesses, possibly leading to vastly different results for the same dataset \autocite{valavi2022SDMperformance}.
Due to those differences, a possible approach is to create an ensemble of multiple models \autocite{araujo2007ensemble}.
The way of combining model predictions can vary, but the goal is to improve total performance by combining the results of all computed models.

% Harmonia axyridis
In order to conduct an iterative modelling approach, a species with sufficient data over the time span of invasion is necessary.
\gls{harm}, also known as the Harlequin ladybird or multicoloured Asian lady beetle, is of the family of the Coccinellidae and has its native origin in Asia \autocite{roy2016harmonia}.
At the time of download, the GBIF dataset for \gls{harm} consisted of 468.462 data points globally, resulting in very sufficient amounts of data (see \ref{ssec:temp_data_change}).
At first widely introduced as a control species against pest aphids, \gls{harm} has turned out to be a highly invasive species reaching an almost global distribution \autocite{brown2008harmonia}.
In America, the species was introduced as early as 1916 (California) and in 1988, first populations outside intended release were found \autocite{chapin1991harmoniaNA}.
Usage of \gls{harm} for biological control in Europe dates back as far as 1990 (France) \autocite{coutanceau2006harmoniaFR}.
First invasive occurrences were confirmed in multiple countries during the early 2000s, including Germany (2000), Belgium (2001), the Netherlands (2002) and the United Kingdom (2003) \autocite{roy2016harmonia}.
The first confirmation in Austria, where it was never used for biological control, was in 2006 \autocite{rabitsch2006harmoniaAT}.
% Maybe put vvv in discussion / results to not already spoil the native niche difference.
It has been shown that all established invasive populations outside of North America have their origin in the first established population in eastern North America, with the European populations being significantly influenced by the used biocontrol strain \autocite{lombaert2010harmoniabridgehead}.

The impact of \gls{harm} on invaded areas is diverse.
In some contexts, the ladybird has been shown to have a negative impact on the diversity and abundance of native ladybird species \autocite{roy2016harmonia}.
Many studies show intra guild predation and direct interspecific competition in favour of \gls{harm} \autocite{pell2008harmoniaIGP}.
This results in a large potential for \gls{harm} to be a significant threat for guild diversity and community structure in its introduced ranges.
It has also been shown that the species feeds on a variety of damaged fruit crops, for example grapes, apples, stone fruit and berry crops, making it a pest in these scenarios \autocite{koch2004harmoniafoodcrop}.
The aggregating behaviour of \gls{harm}, mostly as a strategy for overwintering, is also a cause of disturbance, since private homes and facilities are invaded by large amounts of beetles at a given time \autocite{nalepa2005harmoniahomes}.
In general, \gls{harm} can be concluded to be a species with high impact as an invader, and thus of interest for active research questions.

%existing SDMs for H. axyridis
There have been several publications which modelled and predicted the distribution of \gls{harm}, constrained to certain geographical ranges (i.e. Spain \autocite{ameixa2019harmSDMSpain}, Chile \autocite{alaniz2018harmSDMChile}) or even on global scales \autocite{bidinger2012harmSDMglobalMaxent, poutsma2008harmSDMglobalClimex}.
There has not yet been any model iteration in form of a time series, which is what this thesis aims to add as new insight.
Another goal of this thesis is to look into the limitations of models built early in the invasion process of a species.
By iterating over the years of the invasion, model performance can be evaluated with consideration to the current state of invasion.
In the end, a better understanding of the invasion process of \gls{harm} in Europe and the performance of models trying to capture it should be the result.



\newpage
\section{Materials and Methods} \label{sec:materialsandmethods}
This section elaborates on the Methods used to conduct this research.

\subsection{Datasets} \label{ssec:datasets}
For occurrence data, all global occurrences of \gls{harm} were downloaded from the GBIF database \autocite{GBIFaxyridisdataset}.
All traditional 19 bioclim variables were obtained from the CHELSA V2.1 climatologies dataset \autocite{karger2017CHELSApaper, CHELSAbioclimdataset}, using the 1981-2010 time frame for all years from 2002 to 2010 as well as the MPI-ESM 1.2 ssp370 scenario 2011-2040 for all years from 2011 to 2022.
As additional information, land cover data was used from the Copernicus Land cover Classification dataset \autocite{COPlandcoverdataset}  with yearly resolution for 2002 up to 2020.

\subsection{Data preparation} \label{ssec:datapreparation}
All bioclim and land cover layers were resampled to a matching resolution of 30 arc seconds and cropped to two spatial extents, Europe and the presumed native range referencing (Orlova-Bienkowskaja, Ukrainsky \& Brown, 2015) \autocite{orlova2015harmonia}.

The presence-only points from GBIF were checked for missing values for latitude, longitude, year or coordinate uncertainty and then subset to the afore mentioned spatial extents.
No occurrences after 2022 were used, also no points with a coordinate uncertainty larger than 1 km.
In Europe, the initial cut off year for presences was 1991, since this is the year of invasion according to the EASIN website.
Afterwards, using the library \texttt{CoordinateCleaner}, all remaining data points were again checked for common errors or biases in the respective subset  (tests used: "capitals", "centroids", "duplicates", "equal", "institutions", "outliers", "seas", "zeros").
In addition, all occurrences were checked for their land cover class values in their respective year, removing points in the water or with no data.
In the end, remaining data points prior to 2002 were deemed insignificant and removed from the dataset.
This resulted in a total of 124.746 presence points over all years and areas.
To prepare the data for modelling, pseudoabsences were generated for each year, randomly sampling the area and resampling points in the water or with no data.

To correct for sampling bias in the data, the European and native extents were split into sub extents in order to add additional absences to denser sampled regions.
For this, an algorithm was written which splits a given extent in half and continues to do so with the created subextents until the amount of points in the extents is at most some chosen number.
For the first part of absence generation completely random absences are drawn from the original extent in order to ensure at least some coverage of the whole study area.
In the second step, additional absences are generated for each subextent separately and in relation to the amount of presences inside the respective subextent.
This results in more absences in regions with more presences as well (Fig. \ref{fig:ext_subdiv} A).
The subdivision of the dataset was carried out using all presences in Europe over all years and setting the threshold to be $\leq$ 30\%, leading to subextents converging around the United Kingdom and the Netherlands, which seem to have been sampled very intensely (Fig. \ref{fig:ext_subdiv} B).
For the first and second step of absence generation, absences equal to and twice the amount of presences were generated respectively, resulting in three absences per presence in total.

\begin{figure}[!h]
    \centering
    \includegraphics[width = 0.8\linewidth]{"../../R/figures/ext-subdiv.png"}
    \caption{\label{fig:ext_subdiv} Visualization of the subdivision algorithm and absence generation. Subfigure A shows an example of 30 generated presences (green), subdivided with a threshold of $\leq$ 10 points per subextent. The generated absences are shown in black, with circles indicating the first 30 completely random absences, and asterisks indicating absences generated relative to the amount of presences in a subextent. Subfigure B shows the calculated subextents for the total presences of the dataset, with presences (green) and absences (grey) for 2008 plotted as an example.}
\end{figure}

\subsection{Model building} \label{ssec:modelbuilding}
For each year, the following Models were computed: General Linear (GLM) \autocite{guisan2002glm-gam}, General Additive (GAM) \autocite{guisan2002glm-gam}, Boosted Regression Trees (BRT) \autocite{elith2008brt} and Maximum Entropy (MAXENT) \autocite{phillips2017maxnet}.
A model for a specific year always included all points from past years as well.
The iterative models that were built only use data points from Europe, though there was one model created only with native occurrences and predicted for each year in Europe.
For all used occurrence points after 2020, the land cover data of 2020 was used as a substitute.

Variance inflation factors were used to select the variables used for model building.
For this, a GAM was computed only using Europe data from 2002, using all bioclim and land cover variables.
For land cover variables, a PCA was computed on the relative area of all land cover classes in an 18 km radius around 5000 random data points in Europe, subsequently projecting the occurrence data onto the resulting axes.
The 18 km radius was chosen, since it is the average flight distance determined for \gls{harm} \autocite{jeffries2013flightharmonia}.
PCA axes were included in the model until a cut-off of 80\% of explained variance was reached.
Variance inflation factors were computed for this GAM and the variable with the highest VIF was dropped until none of the remaining variables had a VIF greater than 10.

\subsection{Analysis} \label{ssec:analysis}
All SDM models of each year were evaluated for their accuracy on predicting the occurrences of the following year using the Sensitivity or True Positive Rate (TPR).
The TPR values were used to create a TPR-weighted ensemble of all model predictions, which was again evaluated for its accuracy.
The thresholds used to compute TPR were computed using the library \texttt{PresenceAbsence} while maximizing (sens + spec) / 2. This thresholding was chosen since just maximizing TPR can be achieved trivially by setting the threshold as low as possible, creating a model which predicts everything as suitable (used thresholds in Supplementary Tab. \ref{tab:mod_ths}).
For each year, the occupied niche was computed by running a PCA analysis on the bioclim variables.
The niche was then visualized by plotting a dynamic occurrence density grid for the first two PCA axes \autocite{broennimann2012niche}.
The overlap between each year for Europe and the respective following year was computed, as well as the niche dynamic indices "stability", "expansion" and "unfilling" \autocite{guisan2014nichedyn}.
Niche overlap for the total EU data was also visualized in comparison to the total native niche.
All mentioned niche analyses were conducted using the library \texttt{ecospat} \autocite{dicola2017ecospat}.
The development of TPR over time was tested for correlation with the amount of training data and the niche stability for a given year, using the Pearson correlation test.

\newpage
\section{Results} \label{sec:results}
The analyses mentioned above resulted in findings which shed light on the niche dynamics and predictability of \gls{harm}.

\subsection{Temporal change of data availability} \label{ssec:temp_data_change}
Taking the amount of presence points in Europe and the native range, one can plot the amount of presences available for each year respectively (Fig. \ref{fig:pres_per_year_log}).

\begin{figure}[!h]
    \centering
    \includegraphics[width = 0.9\linewidth]{"../../R/figures/pres-per-year-log.png"}
    \caption{\label{fig:pres_per_year_log} Amount of presence points for \gls{harm} by year and area, using the cleaned dataset (Supplementary Fig. \ref{fig:raw_vs_cleaned_glob}).}
\end{figure}

The resulting figure shows that the amount of data available in the invaded range greatly surpasses the amount in the native range.
With at least 100 presences for any year, the European dataset is definitely sufficient to create SDMs for each year separately, more so if data from previous years is also used.
The exponential increase in observations over time also suggests rapid population growth.
The lack of presence points in early years in the native range was the reason why it was decided to only create one SDM with the native data of all years combined, since it is more likely to provide a complete evaluation of the native niche.

\subsection{Niche dynamic analysis} \label{ssec:niche_dyn_analysis}

\begin{figure}
    \centering
    \includegraphics[width = 0.6\linewidth]{"../../R/figures/as-eu-tot-niche.png"}
    \caption{\label{fig:as_eu_niche} Native (green) and invaded (red) niche of \gls{harm}, shown along the first two axes of a PCA using all bioclim variables (Supplementary Fig. \ref{fig:as_eu_niche_pca}). The blue area indicates overlap between the occupied niches. Dashed and solid lines indicate 50\% and 100\% of the potentially available environment in each area (from background). Grey shading shows the density distribution of the invaded niche.}
\end{figure}

When comparing the total native and invaded niches (so all years included), the niches are clearly different (Fig. \ref{fig:as_eu_niche}).
A niche similarity test produced a p value of $p = 0.32$, leading to an accepted null hypothesis meaning the two niches are not more similar than random (Supplementary Fig. \ref{fig:as_eu_eq_sim}).
Conducting a niche equivalency test, the result was $p = 0.01$ implying highly significant differences between the two niches (Supplementary Fig. \ref{fig:as_eu_eq_sim}).

Looking at the niche only in the invaded range, one can visualize the shift and expansion throughout the years (Fig. \ref{fig:eu_niche_ys}).

\begin{figure}[!h]
    \centering
    \includegraphics[width = \linewidth]{"../../R/figures/eu-niche-ys.png"}
    \caption{\label{fig:eu_niche_ys} The progression of the invaded niche of \gls{harm}, shown along the first two axes of a PCA using all bioclim variables (Supplementary Fig. \ref{fig:eu_years_pca}). Niches of the years (from left to right) 2002, 2012 and 2021 compared to the niche of their following years respectively. Green shows the first year, red shows the year after and blue indicates overlap. Dashed and solid lines indicate 50\% and 100\% of the potentially available environment (from background). Grey shading shows the density distribution of the second year.}

\end{figure}

In addition to a visualization, it makes sense to compute the niche dynamic indices \autocite{guisan2014nichedyn}, as well as the niche overlap (Schoener's D) for each year in order to characterize the niche shift further (Fig. \ref{fig:eu_niche_dyn}).

\begin{figure}[!h]
    \centering
    \includegraphics[width = 0.9\linewidth]{"../../R/figures/eu-niche-dyn.png"}
    \caption{\label{fig:eu_niche_dyn} Niche dynamic indices and overlap (Schoener's D) computed from comparing each year of the invaded niche of \gls{harm} to the following year.}
\end{figure}

The results show that in the first five years of the time series, the invaded niche is still expanding, while afterwards it becomes almost completely stable.
Once stability is reached, \gls{harm} seems to just fill out the rest of the acquired niche, as shown by an increase in niche overlap.

\subsection{SDMs, ensemble and time series}
Variable selection using VIFs resulted in 13 variables, which were used for modelling (Tab. \ref{tab:var_vifs}).

\begin{table}[!h]
    \centering
    \caption{\label{tab:var_vifs} Final 13 variables resulting from variable selection with VIFs, PCA contribution table for all land cover variables in (Supplementary Tab. \ref{tab:lc_contrib}).}
    \centerline{
        \begin{tabular}{l| r*{13}{ c}}
            \csvreader[no head, column count = 14, late after line = \\]{tab-var-vifs.csv}{}{\csvlinetotablerow}
        \end{tabular}
    }
\end{table}

After computing all of the mentioned models (response curves in appendix \ref{sec:response_curves}) and creating a TPR weighted ensemble, one can examine the development over time in more detail (Fig. \ref{fig:modelling_res}).

\begin{figure}[!h]
    \centering
    \includegraphics[width = 0.9\linewidth]{"../../R/figures/modelling-res.png"}
    \caption{\label{fig:modelling_res} SDM predictive performance over time. Subfigure A shows the performance of models created with invasive data up to the year in question in predicting the following year. Subfigure B shows the performance of models created with all the native data in predicting the year in question.}
\end{figure}

The difference in performance between the invaded and native range models is apparent, with the invaded range models achieving a higher TPR on average.
Surprisingly, the invasive data models already perform very well in the first years and continue to do so for all further years.
The native models vary more in their predictive performance, starting off with a TPR similar to that of the invaded range models, then becoming quite chaotic in performance in the years between 2008 and 2015.
After 2015, performance improves again, reaching levels close, but still below the initial performance and that of the final invaded range models.
It is notable, that the GLM almost always performs the best out of all native models.
The model ensemble performed at least better than 50 \% of models used to build the ensemble for almost all years.

Trying to correlate the performance of each model individually to the trend in data amount or the niche stability value lead to statistically insignificant results for all models (Tab. \ref{tab:mod_cor_res}).

\begin{table}[!h]
    \centering
    \caption{\label{tab:mod_cor_res} Results when correlating model TPR to either niche stability or amount of data.}
    \begin{tabular}{>{\bfseries}l r*{5}{ c}}
        \csvreader[no head, column count = 6, late after line = \\]{tab-mod-cor-res.csv}{}{\csvlinetotablerow}
    \end{tabular}
\end{table}

In the end, one can look at the model predictions for the year 2022 and compare them to the observed presences of that year (Fig. \ref{fig:mod_pred}).

\begin{figure}[!h]
    \centering
    \includegraphics[width = \linewidth]{"../../R/figures/2022-mod-pred.png"}
    \caption{\label{fig:mod_pred} Predictions of the invaded (A) and native data (B) ensemble models for the suitability of \gls{harm} in 2022. Suitability shown as colour, black circles indicating presences observed in 2022.}
\end{figure}

The ensemble from the invaded range proves to be more accurate in predicting the presences of 2022, although being rather conservative in its predictions. The native ensemble predicts a much wider range in comparison, reaching much further to the east.


\newpage
\section{Discussion} \label{sec:discussion}
Using a time series of a well monitored species, it was possible to probe into the change in SDM performance over time.
It was shown, that all used SDMs consistently performed very well in predicting \gls{harm} presences of the following year, when trained only on data from the invaded range in Europe.

The better performance of the invaded range models in comparison to the native models is supported by the results of niche analysis, showing that the native and invaded niche differ significantly from each other.
It has already been shown, that the genetic strains of \gls{harm} present in Europe have been strongly influenced by American control strains \autocite{lombaert2010harmoniabridgehead}, so a model using American data might also prove insightful.

The consistent performance of all invaded range models might be in part due to the rather consistent niche stability, though also in years with significant niche expansion (early years) the models are able to perform well.
This could also imply that SDMs are able to perform well in environments without total niche conservatism, but the data shown in this thesis is not sufficient to make this claim.
The lack of impact of data amount on model performance is unsurprising, since all models had a very sufficient amount of data points (\> 15 \autocite{stoa2019SDMdata}).
The native, as well as the invaded range ensemble predict ranges very similar to a study from 2012 \autocite{bidinger2012harmSDMglobalMaxent}, which speaks for the modelling process of this thesis.
Agreement between predictions from 2012 and models created with more recent data also supports the result of niche conservatism over this time period.

The use of TPR as a measure for model performance is not ideal, but was chosen instead of TSS due to better interpretability in the context of an invasive species.
For future work, it would be better to look into different evaluation metrics, as there have been numerous proposed in other works.
It might also be interesting to look into the period of niche expansion of \gls{harm} in more detail, for example with a monthly resolution, since data might be sufficient.

\newpage
\section{Conclusion} \label{sec:conclusion}
The results of this thesis show that SDMs can be a strong tool in predicting the potential niche of an invasive species, also with not completely static niches.
Ensembles proved to be a good way to combine different model predictions, resulting in predictions consistent with previous studies.
Using additional niche analyses provides valuable context to interpret model performance and gives additional insight into the establishment process of \gls{harm}.
It has been shown, that the process used in this thesis can generate new insight into the dynamics and relations at play when dealing with invasive species and predicting their potential impact, with more interesting areas to research further.


\section{Acknowledgements} \label{sec:acknowledgements}
This thesis would not have been possible in this form without the huge amount of freedom given by my supervisor Lauren Talluto.
I was able to just try out my ideas on my own accord, which is something I value highly.
I also want to thank all my friends and relatives, who just listened to me rambling about how cool (or sometimes nerve-wracking) my progress was, even if they only understood half of it.

\newpage
\printbibliography[]
\newpage
\appendix

\section{Visualization of cleaned vs raw dataset}

\begin{figure}[!h]
    \centering
    \includegraphics[width = \linewidth]{"../../R/figures/raw-vs-cleaned-glob.png"}
    \caption{\label{fig:raw_vs_cleaned_glob} Visualization of the cleaned dataset for \gls{harm} in comparison to the total raw dataset. The red and green boxes show the used extents for Europe and the native range respectively, red and green points show the cleaned presence points in their respective extents, while blue points show all points of the raw dataset.}
\end{figure}

\end{document}